%!TEX root = main.tex
\section{Introduction}

The biggest challenge in outdoor Photometric Stereo (PS) remains the fact that the sun, our main light source, moves along a trajectory that nearly lies on a plane throughout the course of the day, leaving the photometric reconstruction under-constrained~\cite{woodham-opteng-80}.
%Researchers have investigated different approaches to address this problem, which include collecting data for several months~\cite{abrams-eccv-12,ackermann-cvpr-12}, waiting for the best time of the year (around summer solstice in the North Hemisphere~\cite{shen-pg-14}) or, more recently, for a day with favorable atmospheric conditions (partly cloudy sky~\cite{holdgeoffroy-iccp-15,holdgeoffroy-3dv-15}).
As previously discussed, one way to circumvent this issue is to wait for partially cloudy days for the effective lighting direction to shift away from the solar plane.
In general, the practitioner has no control over the sun or the atmosphere and is therefore left with two options: ($i$) a potentially long wait for the ideal conditions to arise; or ($ii$) the use of a smarter reconstruction technique that does not rely solely on the photometric cue.

In this chapter, we investigate the second option above. By using machine learning to aggregate prior knowledge on the geometry and reflectance of common classes of objects, we aim to resolve the ambiguities cause by the limited photometric cues available in single-day outdoor PS. To avoid restricting the algorithm to a small family of objects, we build knowledge on local geometry based on the fact that complex surfaces are made up of simpler, piecewise smooth patches. Our approach is further motivated by the fact that material properties are also locally correlated within small surface patches. Working with a broad class of small surface patches keeps the approach general and also facilitates the collection of data for machine learning.

% While non-convolutional deep learning was already used in fully-constrained, indoor PS (mostly to learn a generic, ``inverse BRDF'' function)... 

As we reason in terms of surface patches, a natural choice is to follow a deep learning approach with a novel convolutional neural network (CNN) architecture to automatically learn features from image patches and guide 3D normal estimation. To our knowledge, this is the first CNN to address the problem of under-constrained, single-day outdoor PS.

Our CNN design seeks a balance between fully-calibrated PS (too restrictive) and uncalibrated PS (which introduces further ambiguities). We follow a semi-calibrated outdoor PS approach and only assume that: (1) the object is imaged at roughly predefined times of the day; (2) the sun is unobstructed by clouds at these times; and (3) an orthographic camera facing approximately north (or south). The latter assumption could be relaxed using recent advances on sun position estimation in the wild to automatically detect the camera orientation. One could then train our model on multiple camera calibrations and select the right one according to the detected orientation. Of note, our approach does not require known sun intensities nor complete geolocation data, as in previous work on outdoor PS~\cite{jung-cvpr-15}. Together, our assumptions define an ``8-figure'' subspace for the position of the sun through the year, known as an {\em analemma}, whose shape also varies with geographical location (fig.~\ref{fig:solar-analemma}).

% (PG: that's an assummed/given condition)
% camera (pointing north, orthographic projection)

% analemma (Wikipedia): position of the Sun in the sky over the course of a year, as viewed at a fixed time of day and from a fixed location on the Earth

Therefore, our outdoor PS network is required to learn priors not only on object materials and local geometry (diffuse, specular, smooth, ...), but also priors on lighting (variability in sun intensity and elevation with respect to geolocation and time of year). This knowledge is encoded in the network, which learns to output 3D normals as a non-linear function of input image patches taken throughout a day. This rich prior knowledge allows us to obtain a reconstruction performance that outperforms the current state-of-the-art on real lighting data with only 3 images, which is much lower than the 6-18 typically used by previous work~\cite{yu-iccp-13,jung-cvpr-15}. Our fully trained method will be made publicly available.

%To train our CNN, we realistically render synthetic datasets with a large variety of image patches with different geometry, materials and lighting.

%\todo{Say something about the quality of our results vs state of the art.}

We summarize our contributions as follows:
\begin{itemize}[noitemsep,nolistsep]
	\item a state-of-the-art method for single-day photometric stereo on sunny days that is robust to shadows, specularities and arbitrary but uniform albedo;
    \item a pipeline to generate synthetic images with a mix of Lambertian and glossy materials that captures the diversity of real outdoor lighting.
\end{itemize}



