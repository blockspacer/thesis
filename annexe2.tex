%!TEX root = main.tex
\chapter{Photometric Stereo proof of convexity}     % numérotée
\label{annex2}

In this annex, we prove that the photometric stereo problem is convex.

Under the Lambertian assumption, the photometric stereo problem can be posed as
\begin{equation}
%\arg\min_n \sum_{i \in \mathcal{I}} \left[\sum_j \max\left(0, \mathbf{l}_{ij}^T \mathbf{n} \right) - b_i \right]^2 \quad ,
\arg\min_n \sum_{i \in \mathcal{I}} \left[ \max\left(0, \mathbf{l}_{i}^T \mathbf{n} \right) - b_i \right]^2 \quad ,
\label{eq:init} 
\end{equation}
for a set of images $\mathcal{I}$ having pixel intensities $b_{i}$ for a (albedo-scaled) normal $\mathbf{n}$ lit by its mean light vector $\mathbf{l}_{i}^T$.

This objective function is equivalently formulated in terms of the residual vector \mbox{${\bf r} = [\ldots ,  {r_i} , \ldots]^T$}, where $r_i$ is the residual for the patch's $i$-th image,
\begin{equation}
\arg\min_n  \mathbf{r}^T\mathbf{r} = \arg\min_n \sum_{i \in \mathcal{I}} (r_i)^2 \quad , \mathrm{where} \; r_i = \max( 0, \mathbf{l}_i^T \mathbf{n} ) - b_i \quad,
%\qquad \mathbf{r} = (\mathbf{L}\mathbf{n} - \mathbf{b}), \quad r_i = \mathbf{l}_i^T \mathbf{n} - b_i
\label{eq:defresidual}
\end{equation}
%
which can be rewritten as
%
\begin{equation}
\arg\min_n \sum_{i \in \mathcal{I}} f_i(\mathbf{n})
\qquad, \mathrm{where} \; f_i(\mathbf{n}) = \left( r_i(\mathbf{n}) \right)^2 \quad.
\end{equation}
%

To prove convexity, we'll show that the second derivative of the objective function $\sum_{i \in \mathcal{I}} f_i(\mathbf{n})$ is positive for all ${\bf n}$. First, we find the derivatives of $f_i({\bf n})$:
\begin{align}
\frac{df_i}{d\mathbf{n}} &= 2r_i \left( \frac{\partial r_i}{\partial \mathbf{n}} \right) \quad, \label{eq:firstder}\\
\frac{d^2f_i}{d^2\mathbf{n}} &= 2\left(\frac{\partial r_i}{\partial \mathbf{n}}\right)^2 + 2r_i \frac{\partial ^2 r_i}{\partial ^2 \mathbf{n}} \quad \label{eq:secder}.
\end{align}
We need $\frac{\partial r_i}{\partial \mathbf{n}}$ to continue. Remember that, from eq.~\eqref{eq:defresidual},
\begin{equation}
r_i = \max( 0, \mathbf{l}_i^T \mathbf{n} ) - b_i\\
\quad,
\end{equation}
from which two cases arise due to the $\max$: when the function is constant at 0, or when it equals to $\mathbf{l}_i^T \mathbf{n}$:
\begin{equation}
\frac{\partial r_i}{\partial \mathbf{n}} = \frac{\partial \left[ \max(0, \mathbf{l}_i^T \mathbf{n}) \right]}{\partial \mathbf{n}}  = 
\begin{cases}
\frac{\partial r_i}{\partial \mathbf{n}} = \partial \left[ \mathbf{l}_i^T \mathbf{n} \right]/\partial \mathbf{n} \Rightarrow  \frac{\partial r_i}{\partial \mathbf{n}} = \mathbf{l}_i^T \qquad \text{if} \quad \mathbf{l}_i^T \mathbf{n} > 0        . \\
\frac{\partial [0]}{\partial \mathbf{n}} = 0, \hfill \text{if} \quad \mathbf{l}_i^T \mathbf{n} \leq 0,
\end{cases}\\
\end{equation}
Since $\mathbf{l}_i$ and $\mathbf{n}$ are both vectors ($\in \mathbb{R}^3$ in this case),
\begin{align}
\frac{\partial r_i}{\partial\mathbf{n}} &= \frac{\partial}{\partial\mathbf{n}} (l_1 \cdot n_1 + l_2 \cdot n_2 + l_3 \cdot n_3) = \left[ \frac{\partial r_i}{\partial n_1}, \frac{\partial r_i}{\partial n_2}, \frac{\partial r_i}{\partial n_3} \right] = \mathbf{l}_i^T \quad, \\
\frac{\partial^2 r_i}{\partial\mathbf{n}} &= \frac{\partial}{\partial\mathbf{n}} \left[ \frac{\partial r_i}{\partial n_1}, \frac{\partial r_i}{\partial n_2}, \frac{\partial r_i}{\partial n_3} \right] = \left[ \frac{\partial ^2 r_i}{\partial ^2 n_1}, \frac{\partial ^2 r_i}{\partial ^2 n_2}, \frac{\partial ^2 r_i}{\partial ^2 n_3} \right] = \mathbf{0} \in \mathbb{R}^{1 \times 3} \quad.
\end{align}
Inserting it back in the first derivative eq.~\eqref{eq:firstder},
\begin{equation}
\frac{\partial f_i}{\partial {\bf n}} = \begin{cases}
2 \left( \mathbf{l}_i^T \mathbf{n} - b_i \right) \cdot \mathbf{l}_i^T  \qquad\text{if} \quad \mathbf{l}_i^T \mathbf{n} > 0 \quad, \\
\mathbf{0}  \in \mathbb{R}^{1 \times 3} \hfill  \qquad\text{if} \quad \mathbf{l}_i^T \mathbf{n} \leq 0 \quad,
\end{cases} \\
\label{eq:d1ps}
\end{equation}
and in the second derivative eq.~\eqref{eq:secder},
\begin{equation}
\frac{\partial ^2 f_i}{\partial ^2 {\bf n}} = \begin{cases}
2 \left( \mathbf{l}_i^T \right) ^2 \qquad\text{if} \quad \mathbf{l}_i^T \mathbf{n} > 0 \quad, \\
\mathbf{0}  \in \mathbb{R}^{3 \times 3} \hfill  \qquad\text{if} \quad \mathbf{l}_i^T \mathbf{n} \leq 0 \quad.
\end{cases} \\
\label{eq:d2ps}
\end{equation}

As $\left( \mathbf{l}_i^T\right)^2$ is the result of the outer product $\mathbf{l}_i \mathbf{l}_i^T$, is it a positive semidefinite matrix~\cite{schwerdtfeger1950introduction}. Since the second derivative---or Hessian---of $f_i(\mathbf{n})$ is either null or a positive semidefinite matrix, $f_i(\mathbf{n})$ is hence convex~\cite{schwerdtfeger1950introduction}. The objective function $\sum_{i \in \mathcal{I}} f_i(\mathbf{n})$ is then the sum of convex functions, which gives also a convex function~\cite{schwerdtfeger1950introduction}.

\null\hfill$\square$


%{\bf TODO:} could be simpler to try to prove that $r_i({\bf n})$ is convex first. Then the overall objective function is a sum of squared convex functions -- so it is also convex (check book?)
