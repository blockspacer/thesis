\chapter*{Résumé}                      % ne pas numéroter
\phantomsection\addcontentsline{toc}{chapter}{Résumé} % inclure dans TdM

\begin{otherlanguage*}{french}

Comprendre les images est d'une importance cruciale pour une plétore de tâches, de la composition numérique au ré-éclairage d'une image, en passant par la reconstruction 3D d'objets. Ces tâches permettent aux artistes visuels de réaliser des chef-d'\oe{}uvres ou d'aider des opérateurs à prendre des décisions de façon sécuritaire en fonction de stimulis visuels. Pour beaucoup de ces tâches, les modèles physiques et géométriques que la communauté scientifique a développés donnent lieu à des problèmes mal posés possédant plusieurs solutions, dont généralement une seule est raisonnable. Pour résoudre ces indéterminations, le raisonnement sur le contexte visuel et sémantique d'une scène est habituellement relayé à un artiste ou un expert qui emploie son expérience pour réaliser son travail. Ceci est dû au fait qu'il est généralement nécessaire de raisonner sur la scène de façon globale afin d'obtenir des résultats plausibles et appréciables. Serait-il possible de modéliser l'expérience à partir de données visuelles et d'automatiser en totalité ou en partie ces tâches? Le sujet de cette thèse est celui-ci: la modélisation d'a priori par apprentissage automatique profond pour permettre la résolution de problèmes typiquement mal posés. Plus spécifiquement, nous couvrirons trois axes de recherche, soient: 1) la reconstruction de surface par photométrie, 2) l'estimation d'illumination extérieure à partir d'une seule image et 3) l'estimation de calibration de caméra à partir d'une seule image avec un contenu générique. Ces trois sujets seront abordés avec une perspective axée sur les données. Chacun de ces axes comporte des analyses de performance approfondies et, malgré la réputation d'opacité des algorithmes d'apprentissage machine profonds, nous proposons des études sur les indices visuels captés par nos méthodes. 

\end{otherlanguage*}
