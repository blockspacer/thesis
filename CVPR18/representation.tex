% %!TEX root = main.tex
\section{Geometric camera model}
\label{sec:camera-model}

We first present the geometric camera model used in this chapter. 
%As explained below, our simplified model represents the camera using the focal length $f_\mathrm{px}$, pitch $\theta$ and roll $\psi$ angles.
Under the pinhole camera model, the pixel coordinates $\mathbf{p}_\mathrm{im}$ of a 3D point $\mathbf{p}_\mathrm{w}$ is given by
%
\begin{equation}
\mathbf{p}_{\mathrm{im}} = [\lambda u \; \lambda v \; \lambda]^T = \mathbf{K} \left[\mathbf{R} | \mathbf{t}\right] \left[ \mathbf{p}_{\mathrm{w}} | 1 \right]^T
\end{equation}
%
in homogeneous coordinates, where $\mathbf{K}$ is the camera projection matrix (camera intrinsics), $\mathbf{R}$ and $\mathbf{t}$ are the camera rotation and translation in the world reference frame (camera extrinsics). Simplifying the model further to square pixels, no skew, and image center at the principal point, the projection matrix $\mathbf{K}$ is given by $\mathbf{K} = \mathrm{diag}([f_{\mathrm{px}} \; f_{\mathrm{px}} \; 1])$, where $f_{\mathrm{px}}$ is the focal length in pixels. 

Since the camera parameters are to be estimated from a single image, we first express them as a function of image features. Let us first consider the focal length $f_\mathrm{px}$. Since it has no direct interpretation in the image, we instead estimate the vertical field of view $h_\theta$, a more intuitive measure:
%
\begin{equation}
h_{\theta} = 2 \arctan \left( \nicefrac{ h }{ 2f{_\mathrm{px}} } \right) \,,
\end{equation}
%
where $h$ is the image height.

We next consider the rotation matrix $\mathbf{R}$, which can be parameterized by roll $\psi$, pitch $\theta$, and yaw $\varphi$ angles. There exists no natural reference frame to estimate $\varphi$ (left vs right) from an arbitrary image. Therefore, we constrain the rotation to only pitch and roll components, simplifying the extrinsic rotation matrix to $\mathbf{R} = \mathbf{R}_z(\psi) \mathbf{R}_x(\theta)$. We can use the horizon line as an intuitive representation for these angles. We define the horizon line midpoint $b_{\mathrm{p}}$ as the $y$-coordinate of its intersection with the vertical axis in the image and roll $\psi$ with respect to horizontal.

The midpoint $b_{\mathrm{p}}$ can be derived from $\theta$ and $f_{\mathrm{px}}$ as
%
\begin{equation}
b_{\mathrm{p}} = 2 f_{\mathrm{px}} \tan\theta \,.
\label{eq:horizon_midpoint}
\end{equation}
%
In this image units representation, the top and bottom of the image have coordinates 1 and $-1$ respectively. 

Throughout our work, camera calibration refers to the vertical field of view $h_{\theta}$, pitch $b_{\mathrm{p}}$ and roll $\psi$ from this simplified geometric camera model.
