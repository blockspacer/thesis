%!TEX root = main.tex
\chapter*{Conclusion}         % ne pas numéroter
\phantomsection\addcontentsline{toc}{chapter}{Conclusion} % dans TdM


Throughout this thesis, we show that gathering and analyzing large datasets can bring new insights that can both improve classical methods and build advanced machine learning algorithms. Three main axes of research were presented. First, we proposed a performance prediction framework for the problem of tridimensional surface reconstruction through photometric signal under daylight, a problem we call outdoor Photometric Stereo (PS). Using this framework, we demonstrated that partially cloudy days are more suited to perform PS from photometric signal alone. We also showed that clear days typically does not yield enough photometric information to perform a robust reconstruction. We then presented a learning-based method that augments photometric cues with priors to solve the unstability of PS problem during sunny days. Secondly, we proposed a single image learning-based algorithm for outdoor lighting estimation that is robust to various scene content and exhibits state-of-the-art performance. This method enables automatic photorealistic virtual object insertion and relighting, among others. Lastly, we used a similar approach to create a camera calibration method, which estimates automatically the field or view and the position of the horizon within the image. Applications like image retrieval can thus be automated using this technique and used to streamline compositing operations. All proposed methods work on generic scenes and rely heavily on the priors learned directly from the training data. 

For every project presented, we strove to understand the underlying patterns in the data. We proposed a framework for photometric stereo sensitivity analysis, which can predict reconstruction performance from the sky appearance. We also experimented with our surface reconstruction approach under various configurations to better understand the impact of camera calibration error and the number of input images on the reconstruction performance. Furthermore, the proposed camera calibration estimation method was analyzed through guided backpropagation, which allowed us to better understand the visual cues picked up by the learned model. Even though deep learning models are generally considered \emph{black boxes} by the community, we hope our efforts in explaining the behavior of those machine learning methods have inspired others to continue in this direction.

All methods described in this thesis have a direct application in entertainment, notably by enabling photorealistic virtual object insertion and relighting automatically. After incorporating the lighting estimation and camera calibration techniques into Dimension, the new 3D editing software from Adobe Systems, we received many testimonies of multimedia artists reporting their increased productivity and reduced time needed to create or modify works. Outdoor surface reconstruction can be applied to the 3D scanning of large statues and buildings, where hand-held scanning would take a prohibitive amount of time. The video game and movie industries use similar techniques to digitize actors~\cite{debevec2000acquiring} to produce photorealistic alter-egos for their audience pleasure. Using our proposed approach, the same could be done for large-scale elements like buildings and would allow studios to easily obtain high fidelity models for their creations. Aside from the entertainment industry, this technique allows the preservation of cultural heritage of statues and buildings through digital copies that will stand the test of time. Additionally, the light and camera parameters estimation techniques we developed can also provide additional information to sensors, improving the quality of information provided by decision support systems. Similar systems are currently used as compasses for robots telemetry systems\cite{Ma2017}. Image forensics can also benefit from the ideas proposed in this thesis: detecting shadows orientations (by estimating sun position) and detecting conflicting horizons can help discover image modifications~\cite{Farid2010}. As can be seen, learned priors can be used to perform hard computer vision tasks and has a myriad of applications. It is our hope that this thesis has brought some insights and tools to enable the next generation of deep learned priors to computer vision tasks. 
