%!TEX root = main.tex
\chapter*{Conclusion}         % ne pas numéroter
\phantomsection\addcontentsline{toc}{chapter}{Conclusion} % dans TdM


Throughout this thesis, we show that gathering and analyzing data brings new insights that can both improve classical methods and build advanced machine learning algorithms. Two main axes of research were presented: first, tridimensional surface reconstruction through photometric signal. Secondly, we proposed two algorithms for parameters estimation for both lighting and camera model based on generic scenes and illumination. 

For every project presented, we strove to understand the underlying patterns in the data. We proposed a framework for photometric stereo sensitivity analysis, which can predict reconstruction performance from the sky appearance. This analysis could not have been possible without the hundreds of thousands of sky panoramas we captured. Our second research axis focused on model calibration, first on lighting and then on camera. This work required once again hundreds of thousands images to build the required deep learning models. As can be seen, the techniques proposed in this thesis require tremendous amounts of data. As machine learning techniques continue to improve, especially with the current focus on unsupervised learning, we can expect this amount of data required to train our computer vision algorithms to lower and be less and less structured. 

Computer vision tasks has still a lot to gain from such data-driven approaches in the near future. It is our hope that this thesis has brought some insights and tools to enable the next generation of data-driven methods to computer vision tasks.
